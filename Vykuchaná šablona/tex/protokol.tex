% Výcuc ode mě:
\section*{Výcuc o zpracování}
Rovnice se zadává takto

\begin{equation}
	F(x) = \int_{0}^{x} f(t) \diff t
	\label{eq:rovnice}
\end{equation}

Každý vzorec, obrázek nebo tabulku si pojmenujte pomocí \verb|\label{odkaz}|. Všechny odkazy se v .pdf zobrazí modře a jsou klikatelné.  Jak řešit odkazy v textu:
\begin{itemize}
	\item Na obrázky a tabulky se v textu odkazujte pomocí \verb|\autoref{...}| \autoref{fig:obrazek}.
	
	\item Na literaturu se odkazujte pomocí \verb|\cite{...}|: \cite{Tabulky}.
	
	\item Na rovnice se v textu odkazujte pomocí \verb|\eqref{...}|: \eqref{eq:rovnice}.
	
\end{itemize}			


\begin{figure}[!hbt] %bez této hranaté závorky se obrázek umístí na vrchol stránky
	\centering
	\includegraphics[width=0.3\textwidth]{img/fjfi} %nejdůležitější řádek - obsahuje cestu k obrázku a také jeho relativní šířku k šířce textu
	%všiměte si, že je možné vložit obrázek vnořený ve složce img - tam poté můžete ukládat další obrázky 
	\caption{Popisek obrázku. Převzato z \cite{bib:zadani}} %popisek obrázku, který se zobrazí pod obrázkem
	\label{fig:obrazek} %interní popis obrázku, který použijete k odkazování se
\end{figure}			

\begin{itemize}
	\item Psaní jednotek $\unit{\metre \per \second\squared}$
	\item Projděte si pravidla pro psaní matematických a fyzikálních výrazů: 	\url{http://www.aldebaran.cz/studium/vyrazy.pdf}
	\item Používejte pevnou mezeru \verb|~| tam, kde se nemá zlomit řádek (aby nevznikaly na konci řádku osamocené jednopísmenné předložky): \verb|s~mezerou|. 
	Toto dělá automaticky balíček \verb|encxvlna|, ten je však třeba doinstalovat: \url{https://merlin.fit.vutbr.cz/wiki/index.php/%C4%8Cesk%C3%A1_sazba_v_LaTeXu#Vlnky}
	\hyperref{seznam.cz}{}{}{Seznam}
	
	\item Uměle zalamujte řádky, které přesahují šířku textu a nezalomily se samy: \verb|diagona\-lizovatelnost| = diagona\-lizovatelnost.
	
	\item Používejte desetinnou čárku (český standard), nikoli tečku (anglický standard).
	
	\item Pomlčka jakožto interpunkční znaménko se píše pomocí \verb|--| a mínus je třeba vysázet v matematickém módu (tj. ne \verb|-3 V|, ale \verb|$-3\unit{V}$|): $-3\unit{V}$ 
	
	\item České uvozovky nepište pomocí \verb|,, ... "|, ale příkazem \verb|\uv{...}|, který je součástí zavedeného balíčku \\ \verb|\usepackage[czech]{babel}|: \uv{takto}.
\end{itemize}	


% ----------------------------------------------------------------------
%  Před psaním se důkladně seznamte s Pravidly pro vypracování protokolu!
% ----------------------------------------------------------------------

% ----------------------------------------------------------------------
%  Pracovní úkoly - opište přímo ze zadání
% ----------------------------------------------------------------------
\section{Pracovní úkoly}
\begin{outline}[enumerate]
\1 Úkol jedna byl napsat úkol jedna
\end{outline}
% ----------------------------------------------------------------------
%  Použité pomůcky
% ----------------------------------------------------------------------


\section{Použité přístroje a pomůcky}
\textbf{Pomůcky:} \par 
\textbf{Přístroje:} \par 
\textbf{Programy:} \par

% ----------------------------------------------------------------------
%  Teoretický úvod - vlastními slovy stručne popište fyzikální podstatu měření a uveďte základní vztahy použité ve vypracování
% ----------------------------------------------------------------------
\section{Teoretický úvod}
% ----------------------------------------------------------------------
%  Postup měření - vlastními slovy popište postup měření tak, aby bylo vaše měření reprodukovatelné 
% ----------------------------------------------------------------------
	\section{Postup měření}
			
			

					
			
% ----------------------------------------------------------------------
%  Naměřené hodnoty a samotné vypracování úkolu
% ----------------------------------------------------------------------				
		\section{Vypracování}
					
			
% ----------------------------------------------------------------------
%  Diskuse - obsahuje komentář k jednotlivým výsledkům, porovnání s očekáváním/tabulkovými hodnotami, zdroje především systematických chyb měření, návrh na zlepšení výsledků,...
% ----------------------------------------------------------------------			
\section{Diskuse}
% ----------------------------------------------------------------------
%  Závěr - stručně a jasně shrnout splněné cíle měření, úkoly a výsledky měření
% ----------------------------------------------------------------------
\section{Závěr}


