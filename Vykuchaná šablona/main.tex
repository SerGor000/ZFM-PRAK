% ----------------------------------------------------------------------
%  Základní nastavení dokumentu - musí být na začátku každého tex
%  (pořadí příkazů v této části je důležité!)
% ----------------------------------------------------------------------

%  Typ dokumentu - článek, prezentace aj.
% Standardní základ
%\documentclass[a4paper,10pt]{article} 
%\usepackage[letterpaper]{geometry}
%\geometry{verbose,tmargin=1.5cm,bmargin=2cm,lmargin=2cm,rmargin=2cm}

% Alternativní verze - vhodnější formát papíru (větší hustota textu)
\documentclass[10pt]{scrartcl}
\KOMAoptions{DIV=20} % formát papíru a odsazení od jeho okrajů

%  Kódování výstupu - aby šlo z pdf kopírovat včetně háčků a čárek
\usepackage[T1]{fontenc} 

%  Kódování vstupu - v kódu lze použít háčky a čárky
\usepackage[utf8]{inputenc} 

%  Základní typografická pravidla češtiny/slovenštiny
\usepackage[czech]{babel} 
%\usepackage[slovak]{babel}
% použijte jen jeden z příkazů


%  Lépe vypadající písmo pro T1 kódování
\usepackage{lmodern} 
% je možno zakomentovat, nepoužívá-li se T1 kódování




%  Formátování stránek, empty = odstraní číslování
% \pagestyle{empty}

%  Řádkování
\linespread{1.1}


% ----------------------------------------------------------------------
%  Doplňující balíčky
% ----------------------------------------------------------------------

%  Po desetinné čárce v matematickém módu se nevytvoří mezera
\usepackage{icomma} 

%  Umožňuje pracovat s grafikou
\usepackage{graphicx}

%  Umožňuje použít dva obrázky vedle sebe
\usepackage{subcaption}

%  Pro vkládání obrázků ve formátu eps (např. z gnuplot)
\usepackage{epstopdf} 

%  Automaticky odsadí i první paragraf v každé sekci
\usepackage{indentfirst}

%  Umožňuje rozdělovat obsah na více sloupců
\usepackage{multicol}
\usepackage{booktabs}
\usepackage{pgffor}

%  Umožňuje používat hypertextové odkazy, nastavuje jejich vlastnosti
\usepackage[unicode]{hyperref}


% ----------------------------------------------------------------------
%  Matematika
% ----------------------------------------------------------------------

%  Lepší zobrazování matematiky (rozšíření sum o \limits atd.)
\everymath{\displaystyle}

%  Široké spektrum příkazů pro matematiku
% (Umožní např. psát přes \mathbb{N/R/Q/..} množiny čísel)
\usepackage{amsmath,amssymb}

%  Velikost fontu matematických výrazů v dokumentu lze pro danou
% základního fontu dokumentu upravit pomocí:
% \DeclareMathSizes{X}{Y}{Z}{U} kde:
% X je velikost fontu v dokumentu, pro kterou se matematika upraví
% Y je standartní velikost fontu matematiky
% Z je velikost fontu zmenšených (vnořených výrazů)
% U je velikost fontu ještě více zmenšených (vnořených výrazů).
\DeclareMathSizes{10}{10}{8}{7}

%  Široké spektrum příkazů pro fyziku
\usepackage{physics}

%  Psaní SI jednotek
\usepackage{siunitx}

%  Nám bližší zápis písmene epsilon
\AtBeginDocument{%
%\let\phi\varphi
\let\epsilon\varepsilon
}


% ----------------------------------------------------------------------
%  Pro češtinu/slovenštinu
% ----------------------------------------------------------------------

%  Lokalizace některých názvů do češtiny/slovenštiny
\addto\captionsczech{\renewcommand{\figurename}{Obr.}}
\addto\captionsczech{\renewcommand{\tablename}{Tab.}}
%\addto\captionsczech{\renewcommand{\refname}{Reference}}

\addto\captionsslovak{\renewcommand{\figurename}{Obr.}}
\addto\captionsslovak{\renewcommand{\tablename}{Tab.}}
%\addto\captionsslovak{\renewcommand{\refname}{Reference}}
\renewcommand{\tableautorefname}{Tab.}
\renewcommand{\figureautorefname}{Obr.}
% Odkomentujte následující příkaz, máte-li stažený balíček encxvlna 
% (nutno stáhnout manuálně)
%\usepackage{encxvlna} %vloží nezlomitelné mezery k jednopísmenným

% Nastaví psaní jednotek

\sisetup{inter-unit-product=\ensuremath{{}\cdot{}}}

% Pro úžasně jednoduché citace

\usepackage[
backend=biber,        % if we want unicode and many other features (biber is already by default)
style=iso-numeric, % or iso-numeric for numeric citation method
citestyle=iso-numeric,
sorting=nyvt
]{biblatex}

\addbibresource{./tex/bibliografie.bib}
\DeclareFieldFormat{labelnumberwidth}{[#1]}


% ----------------------------------------------------------------------
%  Soubor s makry
% ----------------------------------------------------------------------
\input{tex/makra}
%  nachází se ve složce /tex/



% ----------------------------------------------------------------------
%  Nastavení odkazů a výsledného pdf
% ----------------------------------------------------------------------
\hypersetup{
colorlinks=true, 
citecolor=blue, 
filecolor=blue, 
linkcolor=blue,
urlcolor=blue, 
pdftitle={\Title},    % title
pdfauthor={\Author},     % author
pdfsubject={Protokol},   % subject of the document
pdfcreator={\Author},   % creator of the document
%     pdfproducer={Producer}, % producer of the document
%     pdfkeywords={keywords}, % list of keywords
pdfnewwindow=true,      % links in new window
}

% ----------------------------------------------------------------------
%  Začátek dokumentu - formátování na výstup
% ----------------------------------------------------------------------
\begin{document}

% Interní proměnné, možno zobrazovat u prezentací, používají se při
% generování pomocí \titlepage apod.
\author{\Author}
\title{\Title}
\date{\Labdate}


% ----------------------------------------------------------------------
%  Hlavička dokumentu
% ----------------------------------------------------------------------

\setlength{\parindent}{0cm}
\begin{multicols}{2}
\textsf{\textbf{\Subject \hspace{10cm} \Institute}\\
%\large  \Title \\[0.5cm]
\textbf{\large{\Title}}}

\begin{tabular}{rlrl}
	 \textsf{Jméno:} & \textbf{\textsf\Author}    &      \textsf{Kolega:} & \textsf{\Coauthor} \\[1.5pt]
	  \textsf{Kruh:} & \textbf{\textsf\Group}     & \textsf{Číslo skup.:} & \textsf{\Circle}   \\[1.5pt]
	\textsf{Měřeno:} & \textbf{\textsf{\Labdate}} &  \textsf{Klasifikace}: & 
\end{tabular}

\begin{flushright}
\vspace*{-0.3cm}
\includegraphics[scale=0.2]{img/fjfi.pdf}
\hspace{0.4cm}
\includegraphics[scale=0.2]{img/cvut.pdf}
\hspace*{1cm}
\end{flushright}
\end{multicols}

\hrule

% ----------------------------------------------------------------------
%  Tělo dokumentu
% ----------------------------------------------------------------------

\setlength{\parindent}{0.5cm}

% ----------------------------------------------------------------------
%  Protokol

\pagenumbering{arabic}  % číslování stránek čísly
% Výcuc ode mě:
\section*{Výcuc o zpracování}
Rovnice se zadává takto

\begin{equation}
	F(x) = \int_{0}^{x} f(t) \diff t
	\label{eq:rovnice}
\end{equation}

Každý vzorec, obrázek nebo tabulku si pojmenujte pomocí \verb|\label{odkaz}|. Všechny odkazy se v .pdf zobrazí modře a jsou klikatelné.  Jak řešit odkazy v textu:
\begin{itemize}
	\item Na obrázky a tabulky se v textu odkazujte pomocí \verb|\autoref{...}| \autoref{fig:obrazek}.
	
	\item Na literaturu se odkazujte pomocí \verb|\cite{...}|: \cite{Tabulky}.
	
	\item Na rovnice se v textu odkazujte pomocí \verb|\eqref{...}|: \eqref{eq:rovnice}.
	
\end{itemize}			


\begin{figure}[!hbt] %bez této hranaté závorky se obrázek umístí na vrchol stránky
	\centering
	\includegraphics[width=0.3\textwidth]{img/fjfi} %nejdůležitější řádek - obsahuje cestu k obrázku a také jeho relativní šířku k šířce textu
	%všiměte si, že je možné vložit obrázek vnořený ve složce img - tam poté můžete ukládat další obrázky 
	\caption{Popisek obrázku. Převzato z \cite{bib:zadani}} %popisek obrázku, který se zobrazí pod obrázkem
	\label{fig:obrazek} %interní popis obrázku, který použijete k odkazování se
\end{figure}			

\begin{itemize}
	\item Psaní jednotek $\unit{\metre \per \second\squared}$
	\item Projděte si pravidla pro psaní matematických a fyzikálních výrazů: 	\url{http://www.aldebaran.cz/studium/vyrazy.pdf}
	\item Používejte pevnou mezeru \verb|~| tam, kde se nemá zlomit řádek (aby nevznikaly na konci řádku osamocené jednopísmenné předložky): \verb|s~mezerou|. 
	Toto dělá automaticky balíček \verb|encxvlna|, ten je však třeba doinstalovat: \url{https://merlin.fit.vutbr.cz/wiki/index.php/%C4%8Cesk%C3%A1_sazba_v_LaTeXu#Vlnky}
	\hyperref{seznam.cz}{}{}{Seznam}
	
	\item Uměle zalamujte řádky, které přesahují šířku textu a nezalomily se samy: \verb|diagona\-lizovatelnost| = diagona\-lizovatelnost.
	
	\item Používejte desetinnou čárku (český standard), nikoli tečku (anglický standard).
	
	\item Pomlčka jakožto interpunkční znaménko se píše pomocí \verb|--| a mínus je třeba vysázet v matematickém módu (tj. ne \verb|-3 V|, ale \verb|$-3\unit{V}$|): $-3\unit{V}$ 
	
	\item České uvozovky nepište pomocí \verb|,, ... "|, ale příkazem \verb|\uv{...}|, který je součástí zavedeného balíčku \\ \verb|\usepackage[czech]{babel}|: \uv{takto}.
\end{itemize}	


% ----------------------------------------------------------------------
%  Před psaním se důkladně seznamte s Pravidly pro vypracování protokolu!
% ----------------------------------------------------------------------

% ----------------------------------------------------------------------
%  Pracovní úkoly - opište přímo ze zadání
% ----------------------------------------------------------------------
\section{Pracovní úkoly}
\begin{outline}[enumerate]
\1 Úkol jedna byl napsat úkol jedna
\end{outline}
% ----------------------------------------------------------------------
%  Použité pomůcky
% ----------------------------------------------------------------------


\section{Použité přístroje a pomůcky}
\textbf{Pomůcky:} \par 
\textbf{Přístroje:} \par 
\textbf{Programy:} \par

% ----------------------------------------------------------------------
%  Teoretický úvod - vlastními slovy stručne popište fyzikální podstatu měření a uveďte základní vztahy použité ve vypracování
% ----------------------------------------------------------------------
\section{Teoretický úvod}
% ----------------------------------------------------------------------
%  Postup měření - vlastními slovy popište postup měření tak, aby bylo vaše měření reprodukovatelné 
% ----------------------------------------------------------------------
	\section{Postup měření}
			
			

					
			
% ----------------------------------------------------------------------
%  Naměřené hodnoty a samotné vypracování úkolu
% ----------------------------------------------------------------------				
		\section{Vypracování}
					
			
% ----------------------------------------------------------------------
%  Diskuse - obsahuje komentář k jednotlivým výsledkům, porovnání s očekáváním/tabulkovými hodnotami, zdroje především systematických chyb měření, návrh na zlepšení výsledků,...
% ----------------------------------------------------------------------			
\section{Diskuse}
% ----------------------------------------------------------------------
%  Závěr - stručně a jasně shrnout splněné cíle měření, úkoly a výsledky měření
% ----------------------------------------------------------------------
\section{Závěr}



% ----------------------------------------------------------------------

% ----------------------------------------------------------------------
%  Literatura

\setcounter{biburllcpenalty}{7000}
\setcounter{biburlucpenalty}{8000}
\nocite{*} 
\printbibliography[title={Použitá literatura},heading=bibnumbered]



% ----------------------------------------------------------------------
%  Příloha

\clearpage 
\pagenumbering{roman}  % číslování stránek písmeny

\setcounter{equation}{0}
%\setcounter{section}{0}
\numberwithin{equation}{section} % případné rovnice budou číslované pod číslem kapitoly

\part*{\LARGE{Příloha}}

\input{tex/apendix}
% ----------------------------------------------------------------------

%\clearpage
				
%\clearpage


\end{document}

% ----------------------------------------------------------------------
%  Konec dokumentu
% ----------------------------------------------------------------------
